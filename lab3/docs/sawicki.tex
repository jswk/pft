\documentclass[a4paper; 12pt]{article}

% global includes
\usepackage[polish]{babel}
\usepackage[utf8]{inputenc}
\usepackage{polski}
\usepackage{courier} %times, kurier
\usepackage{amsmath}
\usepackage{graphicx}
\usepackage{geometry}
\usepackage{indentfirst}
\usepackage{icomma}
\usepackage{booktabs}
\usepackage{hyperref}
\usepackage{pgfplots}
\pgfplotsset{
    compat=1.12,
    width=9cm,
    %yticklabel={\num[round-mode=figures,round-precision=2]{\tick}},
}
%\usepgfplotslibrary{external}
%\tikzexternalize

\usepackage{listings}

% local includes
\usepackage[locale=FR]{siunitx}
\sisetup{per-mode=symbol-or-fraction}

\title{Ćwiczenie III: Modelowanie trajektorii cząstek naładowanych w~polu magnetycznym}
\author{Jakub Sawicki}
\date{\today}

\begin{document}
\renewcommand{\figurename}{Rys.}
\renewcommand{\tablename}{Tab.}
\renewcommand{\abstractname}{Abstrakt}

\maketitle

\section{Ruch cząstki naładowanej w~polu magnetycznym}

Cząstka znajdująca się w~polu magnetycznym opisana jest poniższą funkcją Lagrange'a:
\begin{equation}
    L = \frac{m}{2}\dot{\mathbf{r}}^2+\frac{e}{2}\mathbf{r}\cdot\left( \dot{\mathbf{r}} \times \mathbf{B} \right)
    \label{eq:L}
\end{equation}
W~cylindrycznym układzie współrzędnych z~osią z~skierowaną równolegle do pola, funkcja Hamiltona jest równa:
\begin{equation}
    H=\frac{1}{2m}\left(p_z^2+p_r^2+\frac{1}{r^2}p_\varphi^2\right)-\frac{eB}{2m}p_\varphi+\frac{e^2B^2}{8m}r^2
    \label{eq:H}
\end{equation}
Korzystając z~równań Hamiltona otrzymuje się:
\begin{subequations}
\begin{align}
    \dot{z}&=\frac{p_z}{m}&\dot{p}_z&=0 \\
    \dot{\varphi}&=\frac{p_\varphi}{mr^2}-\frac{eB}{2m}&\dot{p}_\varphi&=0 \\
    \dot{r}&=\frac{p_r}{m}&\dot{p}_r&=\frac{p_\varphi^2}{mr^3}-\frac{e^2rB^2}{4m}
\end{align}
\end{subequations}
Ruch w~kierunku równoległym do pola jest jednostajny i~zostaje zignorowany
(pozostałe współrzędne i~ich pędy też od niego nie zależą).
$p_\varphi=\text{const}$, zostają więc tylko trzy równania do rozwiązania.
Zapisując je w~iteracyjnie:
\begin{subequations}
\begin{align}
    \varphi(t+\Delta t)&=\varphi(t)+\left( \frac{p_\varphi(0)}{mr^2(t)}-\frac{eB}{2m} \right)\Delta t \\
    r(t+\Delta t)&=r(t)+\frac{p_r(t)}{m}\Delta t \\
    p_r(t+\Delta t)&=p_r(t)+\left( \frac{p_\varphi^2(0)}{mr^3(t)}-\frac{e^2rB^2}{4m} \right)\Delta t
\end{align}
\end{subequations}

Dla uproszczenia rachunków reszta parametrów układu $m$, $e$ i $B$ jest równa \num{1}.

\section{Wyniki}

Na wykresach ruch zaczyna się zawsze od punktu $(2,\,0)$.

Uruchomienie przedstawione na Rys.~\ref{fig:data1} pokazuje poprawne zachowanie
się algorytmu dla zadanych danych.
Ruch odbywa się dokładnie wokół zera układu współrzędnych i~energia jest stała
w~czasie.
\begin{figure}
    \centering
    \begin{tikzpicture}[baseline]
        \pgfplotsset{width=7cm}
        \begin{axis}[
                xlabel={czas [\si{\s}]},
                ylabel={energia},
                %ymin=-2, ymax=20, 
        ]
            \addplot[blue] table [x=t, y=E] {0.01_-0.5_1_0_0_2};
        \end{axis}
    \end{tikzpicture}
    \begin{tikzpicture}[baseline]
        \pgfplotsset{width=7cm}
        \begin{axis}[
                xlabel={x},
                ylabel={y},
                %ymin=-2, ymax=20, 
                axis equal,
        ]
            \addplot[red] table [x=x, y=y] {0.01_-0.5_1_0_0_2};
        \end{axis}
    \end{tikzpicture}
    \caption{Rozwiązania dla $\Delta t=\num{0.01}$, $B=\num{-0.5}$,
        $p_\varphi=\num{1}$, $\varphi(0)=\num{0}$, $p_r(0)=\num{0}$
        i~$r(0)=\num{2}$.}
    \label{fig:data1}
\end{figure}

Dla o~połowę mniejszej wartości natężenia pola magnetycznego punkt zatacza
okrąg o~dwa razy większej średnicy, patrz Rys.~\ref{fig:data2}.
Jako, że okres symulacji jest taki sam jak w~poprzednim przypadku, to punkt
zatacza tylko niecałą połowę okręgu, którego obwód jest dwa razy większy~---
a~prędkość punktu niezmieniona.
Tutaj występują wahania energii, jednak dla przyjętego kroku czasowego są one
mniejsze niż \num{0.1} promila.
\begin{figure}
    \centering
    \begin{tikzpicture}[baseline]
        \pgfplotsset{width=7cm}
        \begin{axis}[
                xlabel={czas [\si{\s}]},
                ylabel={energia},
                yticklabel={\num[round-mode=figures,round-precision=5]{\tick}},
        ]
            \addplot[blue] table [x=t, y=E] {0.01_-0.25_1_0_0_2};
        \end{axis}
    \end{tikzpicture}
    \begin{tikzpicture}[baseline]
        \pgfplotsset{width=7cm}
        \begin{axis}[
                xlabel={x},
                ylabel={y},
                %ymin=-2, ymax=20, 
                axis equal,
        ]
            \addplot[red] table [x=x, y=y] {0.01_-0.25_1_0_0_2};
        \end{axis}
    \end{tikzpicture}
    \caption{Rozwiązania dla $\Delta t=\num{0.01}$, $B=\num{-0.25}$,
        $p_\varphi=\num{1}$, $\varphi(0)=\num{0}$, $p_r(0)=\num{0}$
        i~$r(0)=\num{2}$.}
    \label{fig:data2}
\end{figure}

Dodanie obiektowi dodatkowego pędu skierowanego radialnie prowadzi do
odpowiedniej zmiany położenia toru, Rys.~\ref{fig:data3}.
Tutaj błąd energii nie jest większy niż procent.
Można też zaobserwować, że błąd wzrasta w~pobliżu punktu $(0,\,0)$.
\begin{figure}
    \centering
    \begin{tikzpicture}[baseline]
        \pgfplotsset{width=7cm}
        \begin{axis}[
                xlabel={czas [\si{\s}]},
                ylabel={energia},
                yticklabel={\num[round-mode=figures,round-precision=3]{\tick}},
        ]
            \addplot[blue] table [x=t, y=E] {0.01_-0.5_1_0_1_2};
        \end{axis}
    \end{tikzpicture}
    \begin{tikzpicture}[baseline]
        \pgfplotsset{width=7cm}
        \begin{axis}[
                xlabel={x},
                ylabel={y},
                %ymin=-2, ymax=20, 
                axis equal,
        ]
            \addplot[red] table [x=x, y=y] {0.01_-0.5_1_0_1_2};
        \end{axis}
    \end{tikzpicture}
    \caption{Rozwiązania dla $\Delta t=\num{0.01}$, $B=\num{-0.5}$,
        $p_\varphi=\num{1}$, $\varphi(0)=\num{0}$, $p_r(0)=\num{1}$
        i~$r(0)=\num{2}$.}
    \label{fig:data3}
\end{figure}

Przedłużenie symulacji, tak aby punkt zatoczył więcej niż jeden okrąg pozwala
zbadać, czy rozwiązanie wykazuje dryf ze względu na niedokładność obliczeń.
Dla toru położonego symetrycznie wokół punktu $(0,\,0)$ dryf nie występuje, tak
jak i~wahania energii były równe zeru.
Rozmycie okręgu widoczne na Rys.~\ref{fig:data4} ukazującym ten przypadek
wynika z~tego, że punkty choć układają się na okręgu, to tworzą cięciwy, które
mają znaczną wielkość, patrz Rys.~\ref{fig:data41}.
\begin{figure}
    \centering
    \begin{tikzpicture}[baseline]
        \pgfplotsset{width=7cm}
        \begin{axis}[
                xlabel={czas [\si{\s}]},
                ylabel={energia},
                yticklabel={\num[round-mode=figures,round-precision=2]{\tick}},
                xticklabel={\num[round-mode=figures,round-precision=1]{\tick}},
        ]
            \addplot[blue] table [x=t, y=E] {1_-0.5_1_0_0_2};
        \end{axis}
    \end{tikzpicture}
    \begin{tikzpicture}[baseline]
        \pgfplotsset{width=7cm}
        \begin{axis}[
                xlabel={x},
                ylabel={y},
                %ymin=-2, ymax=20, 
                axis equal,
        ]
            \addplot[red] table [x=x, y=y] {1_-0.5_1_0_0_2};
        \end{axis}
    \end{tikzpicture}
    \caption{Rozwiązania dla $\Delta t=\num{1}$, $B=\num{-0.5}$,
        $p_\varphi=\num{1}$, $\varphi(0)=\num{0}$, $p_r(0)=\num{0}$
        i~$r(0)=\num{2}$.}
    \label{fig:data4}
\end{figure}
\begin{figure}
    \centering
    \begin{tikzpicture}[baseline]
        \pgfplotsset{width=7cm}
        \begin{axis}[
                xlabel={x},
                ylabel={y},
                ymin=-2.05, ymax=-1.9, 
                xmin=-0.05, xmax=0.05,
                axis equal,
        ]
            \addplot[red] table [x=x, y=y] {1_-0.5_1_0_0_2};
        \end{axis}
    \end{tikzpicture}
    \caption{Zbliżenie na rozwiązanie dla $\Delta t=\num{1}$, $B=\num{-0.5}$,
        $p_\varphi=\num{1}$, $\varphi(0)=\num{0}$, $p_r(0)=\num{0}$
        i~$r(0)=\num{2}$.}
    \label{fig:data41}
\end{figure}

Sprawienie, że ruch nie jest symetryczny wokół punktu $(0,\,0)$ powoduje
powstanie precesji.
Dla kroku czasowego \num{1} rozwiązanie nie jest stabilne, dlatego na
Rys.~\ref{fig:data5} pokazany jest przypadek kroku czasowego \num{0.1}.
Widać tam precesję rozwiązania wokół punktu $(0,\,0)$, co lepiej widać na
Rys.~\ref{fig:data51}, gdzie linią połączone są co 126 punkty rozwiązania~---
jako że jedno okrążenie ma w~przybliżeniu okres \num{12.6}.
Tor cząstki wykonywać będzie więc dodatkową precesję wokół punktu zero jeżeli
występować będzie ruch w~kierunku radialnym oprócz transwersalnego.
\begin{figure}
    \centering
    \begin{tikzpicture}[baseline]
        \pgfplotsset{width=7cm}
        \begin{axis}[
                xlabel={czas [\si{\s}]},
                ylabel={energia},
                yticklabel={\num[round-mode=figures,round-precision=2]{\tick}},
                xticklabel={\num[round-mode=figures,round-precision=1]{\tick}},
        ]
            \addplot[blue] table [x=t, y=E] {0.1_-0.5_1_0_1_2};
        \end{axis}
    \end{tikzpicture}
    \begin{tikzpicture}[baseline]
        \pgfplotsset{width=7cm}
        \begin{axis}[
                xlabel={x},
                ylabel={y},
                %ymin=-2, ymax=20, 
                axis equal,
        ]
            \addplot[red] table [x=x, y=y] {0.1_-0.5_1_0_1_2};
        \end{axis}
    \end{tikzpicture}
    \caption{Rozwiązania dla $\Delta t=\num{0.1}$, $B=\num{-0.5}$,
        $p_\varphi=\num{1}$, $\varphi(0)=\num{0}$, $p_r(0)=\num{1}$
        i~$r(0)=\num{2}$.}
    \label{fig:data5}
\end{figure}
\begin{figure}
    \centering
    \begin{tikzpicture}[baseline]
        \pgfplotsset{width=7cm}
        \begin{axis}[
                xlabel={x},
                ylabel={y},
                %ymin=-2, ymax=20, 
                axis equal,
        ]
            \addplot[red] table [x=x, y=y] {0.1_-0.5_1_0_1_2_prec};
        \end{axis}
    \end{tikzpicture}
    \caption{Precesja rozwiązania dla $\Delta t=\num{0.1}$, $B=\num{-0.5}$,
        $p_\varphi=\num{1}$, $\varphi(0)=\num{0}$, $p_r(0)=\num{1}$
        i~$r(0)=\num{2}$.}
    \label{fig:data51}
\end{figure}

\section{Podsumowanie}

Analiza rozwiązań wykazała, że rozwiązanie jest dokładne jeśli ruch wykonywany
jest symetrycznie wokół punktu $(0,\,0)$.
Wtedy nie występują wahania energii i~można dość dowolnie manipulować krokiem
czasowym.

Jeżeli ruch nie jest symetryczny (wokół punktu $(0,\,0)$) to występują wahania
energii.
Dla badanych warunków początkowych jednak dobranie kroku czasowego rzędu
\num{0.01} pozwalało zredukować go do skali promila.
Dla takich niesymetrycznych rozwiązań występuje precesja wokół punktu symetrii $(0,\,0)$.
Jej intensywność wzrasta wraz ze zwiększaniem kroku czasowego.
Dla kroku \num{0.01} jest bardzo powolna, jednak zastosowanie kroku \num{0.1}
powoduje, że jej szybkość wzrasta.

%\begin{thebibliography}{9}
    %\bibitem{foster}
        %I.~Foster: \emph{Designing and Building Parallel Programs}
        %\url{http://www.mcs.anl.gov/dbpp/} dostęp 2015.10.14
    %\bibitem{ccpregel}
        %Reza Zadeh: \emph{Lecture 8}
        %\url{http://stanford.edu/~rezab/dao/notes/lec8.pdf} dostęp 2015.11.17
    %\bibitem{realgraphs}
        %\url{https://snap.stanford.edu/data/index.html#web}
%\end{thebibliography}

\end{document}
