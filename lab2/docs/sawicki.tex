\documentclass[a4paper; 12pt]{article}

% global includes
\usepackage[polish]{babel}
\usepackage[utf8]{inputenc}
\usepackage{polski}
\usepackage{courier} %times, kurier
\usepackage{amsmath}
\usepackage{graphicx}
\usepackage{geometry}
\usepackage{indentfirst}
\usepackage{icomma}
\usepackage{booktabs}
\usepackage{hyperref}
\usepackage{pgfplots}
\pgfplotsset{
    compat=1.12,
    width=9cm,
    yticklabel={\num[round-mode=figures,round-precision=2]{\tick}}}
%\usepgfplotslibrary{external}
%\tikzexternalize

\usepackage{listings}

% local includes
\usepackage[locale=FR]{siunitx}
\sisetup{per-mode=symbol-or-fraction}

\title{Ćwiczenie II: Symulacja komputerowa ruchu ciała w~układzie z~więzami}
\author{Jakub Sawicki}
\date{\today}

\begin{document}
\renewcommand{\figurename}{Rys.}
\renewcommand{\tablename}{Tab.}
\renewcommand{\abstractname}{Abstrakt}

\maketitle

\section{Stożek}

W~tym ćwiczeniu rozpatrujemy ruch po powierzchni stożka o~kącie rozwarcia
$2\alpha$ położonego jedną krawędzią na płaskim podłożu.
Dwiema współrzędnymi uogólnionymi będą $z$ --- położenie na osi symetrii stożka
oraz $\varphi$ --- kąt względem pionu.
Stożek znajduje się w~pionowo skierowanym polu grawitacyjnym o~przyspieszeniu $g$.

Ruch punktu materialnego w~tak zdefiniowanych więzach opisany jest równaniami:
\begin{equation}
    \begin{cases}
        \ddot{\varphi}(t) = \frac{g\cos^2{\alpha}}{\sin{\alpha}}\frac{\sin{\varphi}}{z}
            - \frac{2\dot{z}\dot{\varphi}}{z} \\
        \ddot{z}(t) = \sin^2{\alpha}z\dot{\varphi}^2 
            - g\sin{\alpha}\cos^2{\alpha}(1-\cos{\varphi})
    \end{cases}
    \label{eq:ODE}
\end{equation}

Układ taki rozbijany jest na układ czterech równań pierwszego stopnia, które
następnie przekształcane są zgodnie z~metodą Eulera.

Implementacja algorytmu znajduje się
w~\url{https://github.com/jswk/pft/blob/master/lab2/lab2.c}.

\subsection{Wyniki}

W~pierwszej kolejności przeprowadzono obliczenia dla przykładu zadanego
w~treści zadania.
Wykres $\varphi$ oraz $z$ dla tego przypadku przedstawiony jest na
Rys.~\ref{fig:data1}.
Rozwiązanie pokrywa się z~rozwiązaniem przedstawionym w~treści zadania.
Można na tej podstawie wnioskować, że implementacja daje poprawne wyniki
(przynajmniej dla zadanych warunków początkowych).
\begin{figure}[h]
    \centering
    \begin{tikzpicture}
        \pgfplotsset{xmin=0, xmax=10}
        \begin{axis}[
                title={$\alpha=\ang{28};\,\dot{\varphi}=0;\,\varphi=\num{1.1};\,\dot{z}=0;\,z=1$},
                xlabel={czas [\si{\s}]},
                ylabel={$\varphi\,[\text{rad}]$},
                axis y line*=left,
        ]
            \addplot[blue] table [x={t}, y={fi}] {28_0_1.1_0_1.data};
        \end{axis}
        \begin{axis}[
                axis y line*=right,
                ylabel={$z\,[\text{m}]$},
        ]
            \addplot[red] table [x={t}, y={z}] {28_0_1.1_0_1.data};
        \end{axis}
    \end{tikzpicture}
    \caption{Rozwiązanie dla warunków początkowych z opisu zadania.
        $\alpha=\ang{28};\,\dot{\varphi}=0;\,\varphi=\num{1.1};\,\dot{z}=0;\,z=1$}
    \label{fig:data1}
\end{figure}

Najciekawsze rozwiązania powstają dla położeń blisko ostrza stożka.
Poniższe przypadki eksplorują te sytuacje.

Poprawność rozwiązania można dodatkowo sprawdzić jak zachowywać się będzie
punkt materialny jeżeli wpuszczony zostanie do stożka o mniejszym kącie
rozwarcia z takimi samymi warunkami początkowymi jak dla stożka z treści
zadania.
Ze względu na mniejszą rozwartość drgania powinny być o większej
częstotliwości, co rzeczywiście jest obserwowane dla takiego rozwiązania, patrz
Rys.~\ref{fig:data2}.
\begin{figure}[h]
    \centering
    \begin{tikzpicture}
        \pgfplotsset{xmin=0, xmax=10}
        \begin{axis}[
                title={$\alpha=\ang{10};\,\dot{\varphi}=0;\,\varphi=\num{1};\,\dot{z}=0;\,z=1$},
                xlabel={czas [\si{\s}]},
                ylabel={$\varphi\,[\text{rad}]$},
                axis y line*=left,
        ]
            \addplot[blue] table [x={t}, y={fi}] {10_0_1_0_1.data};
        \end{axis}
        \begin{axis}[
                axis y line*=right,
                ylabel={$z\,[\text{m}]$},
        ]
            \addplot[red] table [x={t}, y={z}] {10_0_1_0_1.data};
        \end{axis}
    \end{tikzpicture}
    \caption{Rozwiązanie dla $\alpha=\ang{10};\,\dot{\varphi}=0;\,\varphi=\num{1};\,\dot{z}=0;\,z=1$}
    \label{fig:data2}
\end{figure}

\subsection{Ruch w stronę ostrza}
Następnie przeliczone zostały przykłady, w których punkt porusza się w stronę
ostrza stożka.
Nie może on poruszać się dokładnie w kierunku ostrza~--- przebije się wtedy
na stronę ujemnych wartości $z$ co znajduje się poza obszarem rozwiązań naszego
modelu (rozwiązania takie zachowują się tak, jak gdyby stożek był położony pod
powierzchnią, jeśli dla dodatnich $z$ znajduje się nad nią).

Przeliczone zostały więc dwa przykłady, w których oprócz prędkości w kierunku
ostrza punkt ma nadaną niewielką prędkość transwersalną.
Przedstawione zostały na Rys.~\ref{fig:data3}.
Na wykresach widać, że punkt zbliża się do ostrza na niewielką odległość gdzie
za względu na niezerowy moment pędu osiąga dużą prędkość transwersalną.
Dokonuje kilku obrotów wokół ostrza i odbija się od niego.
\begin{figure}[h]
    \centering
    \begin{tikzpicture}[baseline]
        \pgfplotsset{xmin=0, xmax=1.5, width=7cm}
        \begin{axis}[
                title={$\alpha=\ang{10};\,\dot{\varphi}=\num{0.1};\,\varphi=\num{0};\,\dot{z}=-1;\,z=1$},
                xlabel={czas [\si{\s}]},
                ylabel={$\varphi\,[\text{rad}]$},
                axis y line*=left,
                ymin=-2, ymax=20, 
        ]
            \addplot[blue] gnuplot [raw gnuplot] {
                plot '10_0.1_0_-1_1.data' u 1:2 every 50;
            };
        \end{axis}
        \begin{axis}[
                axis y line*=right,
                yticklabels={},
                %ylabel={$z\,[\text{m}]$},
        ]
            \addplot[red] gnuplot [raw gnuplot] {
                plot '10_0.1_0_-1_1.data' u 1:3 every 50;
            };
        \end{axis}
    \end{tikzpicture}
    \begin{tikzpicture}[baseline]
        \pgfplotsset{xmin=0, xmax=1.5, width=7cm}
        \begin{axis}[
                title={$\alpha=\ang{10};\,\dot{\varphi}=\num{1};\,\varphi=\num{0};\,\dot{z}=-1;\,z=1$},
                xlabel={czas [\si{\s}]},
                %ylabel={$\varphi\,[\text{rad}]$},
                yticklabels={},
                axis y line*=left,
                ymin=-2, ymax=20, 
        ]
            \addplot[blue] gnuplot [raw gnuplot] {
                plot '10_1_0_-1_1.data' u 1:2 every 50;
            };
        \end{axis}
        \begin{axis}[
                axis y line*=right,
                ylabel={$z\,[\text{m}]$},
        ]
            \addplot[red] gnuplot [raw gnuplot] {
                plot '10_1_0_-1_1.data' u 1:3 every 50;
            };
        \end{axis}
    \end{tikzpicture}
    \caption{Rozwiązania dla punktu poruszającego się w stronę ostrza stożka z
        różnymi prędkościami transwersalnymi.}
    \label{fig:data3}
\end{figure}

\subsection{Bardzo rozwarty stożek}
Obliczone zostały również rozwiązania dla stożka o kątach rozwarcia
$2\alpha=\ang{90}$ oraz $2\alpha=\ang{170}$.
Rozwiązania są pokazane na Rys.~\ref{fig:data4}.
\begin{figure}[h]
    \centering
    \begin{tikzpicture}[baseline]
        \pgfplotsset{xmin=0, xmax=4, width=6cm}
        \begin{axis}[
                title={$\alpha=\ang{45};\,\dot{\varphi}=\num{0.1};\,\varphi=\num{0};\,\dot{z}=-2;\,z=1$},
                xlabel={czas [\si{\s}]},
                ylabel={$\varphi\,[\text{rad}]$},
                axis y line*=left,
                %ymin=-2, ymax=20, 
        ]
            \addplot[blue] gnuplot [raw gnuplot] {
                plot '45_0.1_0_-2_1.data' u 1:2 every 50;
            };
        \end{axis}
        \begin{axis}[
                axis y line*=right,
                %yticklabels={},
                %ylabel={$z\,[\text{m}]$},
        ]
            \addplot[red] gnuplot [raw gnuplot] {
                plot '45_0.1_0_-2_1.data' u 1:3 every 50;
            };
        \end{axis}
    \end{tikzpicture}
    \begin{tikzpicture}[baseline]
        \pgfplotsset{xmin=0, xmax=40, width=6cm}
        \begin{axis}[
                title={$\alpha=\ang{85};\,\dot{\varphi}=\num{0.5};\,\varphi=\num{0};\,\dot{z}=-2;\,z=1$},
                xlabel={czas [\si{\s}]},
                %ylabel={$\varphi\,[\text{rad}]$},
                %yticklabels={},
                axis y line*=left,
                %ymin=-2, ymax=20, 
        ]
            \addplot[blue] gnuplot [raw gnuplot] {
                plot '85_0.5_0_-2_1.data' u 1:2 every 50;
            };
        \end{axis}
        \begin{axis}[
                axis y line*=right,
                ylabel={$z\,[\text{m}]$},
        ]
            \addplot[red] gnuplot [raw gnuplot] {
                plot '85_0.5_0_-2_1.data' u 1:3 every 50;
            };
        \end{axis}
    \end{tikzpicture}
    \caption{Rozwiązania dla punktu poruszającego się w stronę ostrza stożków o
        kątach rozwarcia \ang{90} i \ang{170}.}
    \label{fig:data4}
\end{figure}

Dla kąta rozwarcia \ang{90} punkt omija ostrze wykonując pełny obrót wokół niego.

Dla kąta rozwarcia \ang{170} punkt wchodzi daleko na lekko nachyloną część
stożka i powoli hamuje.
Następnie wraca na dół mijając ostrze w większej odległości.

%\begin{thebibliography}{9}
    %\bibitem{foster}
        %I.~Foster: \emph{Designing and Building Parallel Programs}
        %\url{http://www.mcs.anl.gov/dbpp/} dostęp 2015.10.14
    %\bibitem{ccpregel}
        %Reza Zadeh: \emph{Lecture 8}
        %\url{http://stanford.edu/~rezab/dao/notes/lec8.pdf} dostęp 2015.11.17
    %\bibitem{realgraphs}
        %\url{https://snap.stanford.edu/data/index.html#web}
%\end{thebibliography}

\end{document}
