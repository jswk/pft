\documentclass[a4paper; 12pt]{article}

% global includes
\usepackage[polish]{babel}
\usepackage[utf8]{inputenc}
\usepackage{polski}
\usepackage{courier} %times, kurier
\usepackage{amsmath}
\usepackage{graphicx}
\usepackage{geometry}
\usepackage{indentfirst}
\usepackage{icomma}
\usepackage{booktabs}
\usepackage{hyperref}

\usepackage{listings}

\graphicspath{{plt/}}

% local includes
\usepackage[locale=FR]{siunitx}
\sisetup{per-mode=symbol-or-fraction}

\title{Ćwiczenie IB: Wahadło matematyczne}
\author{Jakub Sawicki}
\date{\today}

\begin{document}
\renewcommand{\figurename}{Rys.}
\renewcommand{\tablename}{Tab.}
\renewcommand{\abstractname}{Abstrakt}

\maketitle

\section{Wahadło matematyczne}

Wahadło matematyczne to punkt materialny zawieszony na nieważkiej
i~nierozciągliwej nici o~długości $R$ w~jednorodnym polu grawitacyjnym
o~przyspieszeniu $g$ i~skierowanym w~dół.
Układ taki ma tylko jeden stopień swobody a~mianowicie kąt $\varphi$
określający położenie punktu.
$\varphi=0$ dla dolnego punktu równowagi wahadła.

Równanie ruchu dla takiego układu ma postać
\begin{equation}
    R\ddot{\varphi}+g\sin{\varphi} = 0 \text{ .}
    \label{eq:eqofmotion}
\end{equation}
Można je rozwiązać analitycznie dla niewielkich $\varphi$, wtedy rozwiązanie ma postać
\begin{equation}
    \varphi(t) = A \sin(\beta t + \alpha) \text{ ,}
    \label{eq:analyticsolution}
\end{equation}
nie jest to jednak rozwiązanie poprawne dla większych $\varphi$.


\section{Rozwiązanie numeryczne}

W~celu rozwiązania równania różniczkowego wyższego rzędu niż pierwszego należy
je sprowadzić do układu równań pierwszego stopnia.
\begin{equation}
    \begin{cases}
        \dot{\omega}(t) = - \beta^2\sin{\varphi(t)} \\
        \dot{\varphi}(t) = \omega(t)
    \end{cases}
    \label{eq:1storderODE}
\end{equation}

Rozwiązanie tego układu równań będziemy prowadzić na pewnej zdyskretyzowanej
siatce co wymusza wprowadzenie pewnego kroku czasowego (w~tym przypadku
stałego) oraz zastosowanie różnic skończonych do wyrażenia przybliżeń różniczek.
\begin{equation}
    \begin{cases}
        d\omega(t) = - \beta^2 \sin{\varphi(t)} dt \\
        d\varphi(t) = \omega(t) dt
    \end{cases}
\end{equation}
\begin{equation}
    \begin{cases}
        \Delta\omega(t) = - \beta^2 \sin{\varphi(t)} \Delta t \\
        \Delta\varphi(t) = \omega(t) \Delta t
    \end{cases}
\end{equation}
\begin{equation}
    \begin{cases}
        \omega(t_{i+1}) = \omega(t_{i}) - \beta^2 \sin{\varphi(t_i)} \Delta t \\
        \varphi(t_{i+1}) = \varphi(t_i) + \omega(t_{i+1}) \Delta t
    \end{cases}
\end{equation}
Obliczenia dla kolejnych momentów prowadzić najwygodniej jest sekwencyjnie.

\subsection{Wyniki}

Przeprowadzone zostały obliczenia dla różnych początkowych wychyleń.
Przykładowe wyniki pokazane są na Rys.~\ref{fig:differentfi0}.
Widać, że wraz ze zwiększaniem amplitudy drgań wydłuża się też okres.

\begin{figure}[h]
    \centering
    \includegraphics[width=.5\textwidth]{different_fi0}
    \caption{Wykres pokazuje przykładowe rozwiązania dla różnych początkowych
        kątów wychyleń.
        Zaobserwować można zwiększanie się okresu ze zwiększaniem amplitudy drgań.}
    \label{fig:differentfi0}
\end{figure}

Zależność ta została dokładniej zbadana.
Na Rys.~\ref{fig:periodchange} pokazana jest zależność okresu od amplitudy
początkowej.
Rośnie ona wraz ze wzrostem początkowego wychylenia.

\begin{figure}[h]
    \centering
    \includegraphics[width=.5\textwidth]{period_change}
    \caption{Na wykresie pokazana została zależność okresu drgań od amplitudy początkowej.}
    \label{fig:periodchange}
\end{figure}

Wykres Rys.~\ref{fig:smallstepenergy} pokazuje zmianę energii całkowitej
w~funkcji czasu.
Mimo że powinna ona być stała w~trakcie ruchu, to w~symulacji numerycznej można
zaobserwować jej odstępstwa od stałej wartości.
Wraz ze zmniejszaniem kroku czasowego odchylenia te zmniejszają się.

\begin{figure}[h]
    \centering
    \includegraphics[width=.5\textwidth]{small_step_energy}
    \caption{Wykres pokazuje wartości energii całkowitej układu dla różnych
        kroków czasowych.}
    \label{fig:smallstepenergy}
\end{figure}

Miara błędu metody dla danego kroku czasowego została przybliżona poprzez
amplitudę wahania się wartości energii całkowitej.
Wynik takiej analizy pokazany jest na Rys.~\ref{fig:error}.
Wartość energii całkowitej oscyluje z~częstotliwością dwa razy większą niż ruch
wahadła.
Z~tego względu do analizy wzięty został jedynie półokres wahadła, z~tego wynika
chaos dla wartości kroku czasowego powyżej 0,15 gdzie metoda przestaje być stabilna.
W~zakresie długości kroków czasowych, kiedy metoda jest stabilna zależność
pomiędzy krokiem czasowym i~miarą błędu jest liniowa.

\begin{figure}[h]
    \centering
    \includegraphics[width=.5\textwidth]{error}
    \caption{Zależność miary błędu (amplitudy wahań energii całkowitej) od
        długości kroku czasowego.
        Dla dużych wartości kroku czasowego metoda traci stabilność co widać po
        chaotycznym wykresie.}
    \label{fig:error}
\end{figure}

%\begin{thebibliography}{9}
    %\bibitem{foster}
        %I.~Foster: \emph{Designing and Building Parallel Programs}
        %\url{http://www.mcs.anl.gov/dbpp/} dostęp 2015.10.14
    %\bibitem{ccpregel}
        %Reza Zadeh: \emph{Lecture 8}
        %\url{http://stanford.edu/~rezab/dao/notes/lec8.pdf} dostęp 2015.11.17
    %\bibitem{realgraphs}
        %\url{https://snap.stanford.edu/data/index.html#web}
%\end{thebibliography}

\end{document}
